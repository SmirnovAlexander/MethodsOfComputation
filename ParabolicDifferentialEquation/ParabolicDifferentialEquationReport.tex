\documentclass[a4paper,12pt]{article}

\usepackage{cmap}					% Document search.
\usepackage{graphicx}               % Images support.
\usepackage[none]{hyphenat}         % No line break.
\usepackage[T2A]{fontenc}			
\usepackage[utf8]{inputenc}			
\usepackage[english,russian]{babel}	
\usepackage{amsmath}
\author{Александр Смирнов, 341 группа}
\title{Метод сеток для решения уравнения параболического типа}
\date{\today}

\begin{document} 

\maketitle

\section{Постановка задачи}

    Рассмотрим уравнение параболического типа

    $$\frac{\partial u}{\partial t}=L u + f(x, t), \quad 0<x<1, \quad 0<t \leq T, \quad f(x, t) \in C_{[0,1] \times[0, T]} $$

    удовлетворяющее начальному условию
    $$ u(x, 0)=\varphi(x), \quad 0 \leq x \leq 1, \quad \varphi(x) \in C_{[0,1]} $$ 

    и граничным условиям
    $$\alpha_{1}(t) u(0, t)-\alpha_{2}(t) \frac{\partial u}{\partial x}(0, t)=\alpha(t)$$
    $$\alpha_{1}(t) \alpha_{2}(t) \geq 0, \quad\left|\alpha_{1}(t)\right|+\left|\alpha_{2}(t)\right|>0, \quad 0 \leq t \leq T $$
    $$\beta_{1}(t) u(1, t)+\beta_{2}(t) \frac{\partial u}{\partial x}(1, t)=\beta(t)$$ 
    $$\beta_{1}(t) \beta_{0}(t) \geq 0 \quad\left|\beta_{1}(t)\right|+\left|\beta_{0}(t)\right|>0, \quad 0 \leq t \leq T$$

    $L u$ может иметь один из двух видов


    $$L u=
    \begin{cases}

        \text { a) } a(x, t) \frac{\partial^{2} u}{\partial x^{2}}+b(x, t) \frac{\partial u}{\partial x}+c(x, t) u, \quad a(x, t) \in C_{[0,1] \times[0, T]},\quad a(x, t) \geq a_{0}>0, 
        \\
        \text { b) } \frac{\partial}{\partial x}\left(p(x) \frac{\partial u}{\partial x}\right)+b(x, t) \frac{\partial u}{\partial x}+c(x, t) u, \quad p(x) \in C_{[0,1]}, \quad p(x) \geq p_{0}>0, \quad 0<x<1, 
        \\
        \quad b(x, t) \in C_{[0,1] \times[0, T]}, \quad c(x, t) \in C_{[0,1] \times[0, T]}, \quad c(x, t) \leq 0.

    \end{cases}$$

    Нам нужно найти в $\bar{D}=[0,1] \times[0, T]$ решение $u(x, t)$.

\section{Построение сетки}

    Разобьём отрезок $[0, 1]$ на $N$ равных частей и отрезок $[0, T]$ на $M$ равных частей. Построим сетку узлов $\overline{\omega_{h \tau}}=\left\{\left(x_{i}, t_{k}\right), i=\overline{0, N} ; k=\overline{0, M}\right\}$. 

    Приближённое решение будем искать в виде таблицы значений в точках сетки.

    Используя аппроксимации дифференциальных выражений разностными, заменяем оператор $L$ разностным оператором

    $$L_{h} u_{i}^{k}=
    \begin{cases}

        \text { a) } a\left(x_{i}, t_{k}\right) \frac{u_{i+1}^{k}-2 u_{i}^{k}+u_{i-1}^{k}}{h^{2}}+b\left(x_{i}, t_{k}\right) \frac{u_{i+1}^{k}-u_{i-1}^{k}}{2 h}+c\left(x_{i}, t_{k}\right) u_{i}^{k}
\       \\
        \text { b) } p_{i+\frac{1}{2}} \frac{u_{i+1}^{k}-u_{i}^{k}}{h^{2}}-p_{i-\frac{1}{2}} \frac{u_{i}^{k}-u_{i-1}^{k}}{h^{2}}+b\left(x_{i}, t_{k}\right) \frac{u_{i+1}^{k}-u_{i-1}^{k}}{2 h}+c\left(x_{i}, t_{k}\right) u_{i}^{k}
       \\
    \end{cases}$$

\section{Явная разностная схема}

    Будем аппроксимировать уравнение в узле $(x_i, t_{k-1})$

    $$\frac{u_{i}^{k}-u_{i}^{k-1}}{\tau}=L_{h} u_{i}^{k-1}+f\left(x_{i}, t_{k-1}\right), \quad i=\overline{1, N-1}, \quad k=\overline{1, M}$$

    Каждое уравнение будет содержать решения только в четырёх точках $u_{i-1}^{k-1}, u_{i}^{k-1}, u_{i+1}^{k-1}, u_{i}^{k}$.

    Из начального условия получим
    $$u_{i}^{0}=\varphi\left(x_{i}\right), \quad i=\overline{0, N}$$

    Аппроксимируем граничные условия
    $$\begin{array}{c}
\alpha_{1}\left(t_{k}\right) u_{0}^{k}-\alpha_{2}\left(t_{k}\right) \frac{-3 u_{0}^{k}+4 u_{1}^{k}-u_{2}^{k}}{2 h}=\alpha\left(t_{k}\right) \\
\beta_{1}\left(t_{k}\right) u_{N}^{k}+\beta_{2}\left(t_{k}\right) \frac{3 u_{N}^{k}-4 u_{N-1}^{k}+u_{N-2}^{k}}{2 h}=\beta\left(t_{k}\right) \\
k=\overline{1, M}
\end{array}$$

    Решение исходной задачи свелось к решению последних 4-х уравнений.

\section{Схема с весами}

    Пусть $\sigma$ -- вещественный параметр.

    Рассмотрим однопараметрическое свойство разностных схем
    $$\frac{u_{i}^{k}-u_{i}^{k-1}}{\tau}=L_{h}\left(\sigma u_{i}^{k}+(1-\sigma) u_{i}^{k-1}\right)+f\left(x_{i}, \overline{t_{k}}\right), \quad i=\overline{1, N-1} \quad k=\overline{1, M}$$

    Из начального условия будем иметь
    $$u_{i}^{0}=\varphi\left(x_{i}\right), \quad i=\overline{0, N}$$

    Производные в краевых условиях аппроксимируем с первым порядком
    $$\begin{array}{c}
        \alpha_{1}\left(t_{k}\right) u_{0}^{k}-\alpha_{2}\left(t_{k}\right) \frac{u_{1}^{k}-u_{0}^{k}}{h}=\alpha\left(t_{k}\right) \\
        \beta_{1}\left(t_{k}\right) u_{N}^{k}+\beta_{2}\left(t_{k}\right) \frac{u_{N}^{k}-u_{N-1}^{k}}{h}=\beta\left(t_{k}\right)
    \end{array}$$

    Так как к моменту определения решения на $k$-ом слое решение на предыдущем слое уже известно, перепишем систему:
    $$\sigma L_{h} u_{i}^{k}-\frac{1}{\tau} u_{i}^{k}=G_{i}^{k}$$

    где
    $$G_{i}^{k}=-\frac{1}{\tau} u_{i}^{k-1}-(1-\sigma) L_{h} u_{i}^{k-1}-f\left(x_{i}, \overline{t_{k}}\right), \quad i=\overline{1, N-1}, \quad k=\overline{1, M}$$

    Граничные условия приведём к виду
    $$\begin{array}{l}
        -B_{0} u_{0}^{k}+C_{0} u_{1}^{k}=G_{0}^{k} \\
        A_{N} u_{N-1}^{k}-B_{N} u_{N}^{k}=G_{N}^{k}
    \end{array}$$

    Таким образом, на каждом $k$-ом слое в данном случае приходится решать систему $(N+1)$ порядка с трехдиагональной матрицей следующего вида
    $$\begin{aligned}
        -B_{0} u_{0}^{k} &+C_{0} u_{1}^{k}=G_{0}^{k} \\
        A_{i} u_{i-1}^{k}-B_{i} u_{i}^{k} &+C_{i} u_{i+1}^{k}=G_{i}^{k}, \quad i=\overline{1, N-1} \\ A_{N} u_{N-1}^{k}-B_{N} u_{N}^{k} &=G_{N}^{k}, \\
        k=\overline{1, M}
    \end{aligned}$$

    Для решения системы будем использовать метод прогонки.

\end{document} 
